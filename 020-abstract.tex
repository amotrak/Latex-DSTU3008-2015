\begin{abstractpage}
    \begin{abstract}{ukrainian}
        \par
        % FIX: page counter should stop at appendices as per DSTU
        Звіт про КР: \pageref{LastPage} с., \TotalValue{totaltables} табл., \TotalValue{totalfigures} рис., \TotalValue{totallistings} ліст., \TotalValue{appendixchapters} дод., \thetotalbibliography\ джерел.
        
        \bigskip
        \MakeUppercase{Atmega328p, датчики, термістор, термопара, NTC, DS18B20, DHT11, DHT22, HTU21D, цифрові, аналогові, proteus}
        \bigskip

        \textbf{Об'єкт дослідження} -- цифрові та аналогові датчики температури.
        
        \textbf{Метою} цієї роботи порівняння характеристик аналогових та цифрових датчиків температури, а також розробка програми для їх інтеграції з мікроконтролером.

        \textbf{Актуальність} роботи полягає у тому, що в сучасному світі неможливо розробляти електронні пристрої, не вимірюючи, при цьому температури їхнього мікропроцесора, нагрівального елементу, пристроїв збереження інформації (жорстких та SSD дисків), а також, профільних рішень клімат-контролю та ``розумної'' теплиці. А отже, потрібно вибирати правильні датчики, в залежності від вимог продукту.

        \textbf{Структура роботи}. Пропонована курсова робота складається із вступу, двох розділів -- теоретичної (вибір датчиків та їх загальний огляд) та практичної частини (проєктування схеми у САПР Proteus, написання програми та збірка реального макету, графіки), висновків до кожної з них, загальних висновків та списку літератури.

        Було зібрано пристрій, на базі Arduino UNO R3, до
        якого під'єднано декілька найцікавіших, на мій погляд, датчиків. Уся
        інформація виводитися на LCD дисплей, дані для графіків
        відправляти за допомогою Serial інтерфейсу на комп'ютер, де далі будуються графіки та порівнюються результати вимірювань.
    \end{abstract}
    
    % \begin{abstract}{english}
    %     \par
    %     Report on the implementation of the KR: 29 pages, 7 figures, 1 table, 8 sources, 1 addition.

    %     Keywords: PLUG AND PLAY, SCSI BUS, EGA, VGA, SVGA, PIT TIMER, ISA, CSN, LFSR.

    %     Purposes described: Plug and Play principle of operation and implementation in the PCI bus; use of the PIT timer, description of its channels, possibilities from the side of the minimum and maximum interval for which the timer can be programmed; internal SCSI bus architecture, speed and actual purpose; video adapters reviewed and compared
        
    %     An algorithm and a program were developed, which in the background reacts to requests to the hard disk and plays the specified melody.
    % \end{abstract}
\end{abstractpage}

\clearpage
