\AddAssociatedCounters{chapter}{appendixchapters}

\chapter{Код програми}
\thispagestyle{headings}

\begin{lstlisting}[style=cpp-small, caption=Базовий клас температурного датчика, label=lst:sensor_base]
enum class SensorType {
    kNTC,
    kDS18,
    kDHT11,
    kDHT22,
    kHTU21,
    kThermocouple
};

class Sensor {
    public:
    Sensor(const char *label, uint16_t interval): label_(label), interval_(interval) {}
    ~Sensor() {};
    const char* getLabel() const { return label_; };
    uint16_t getInterval() const { return interval_; };
    void getCachedInfo(float *temp, float *humidity) {
        *temp = temp_;
        *humidity = humidity_;
    }

    virtual bool begin() { return true; };
    virtual bool requestTemp() { return true; };
    virtual SensorType getType() const = 0;
    virtual void getInfo(float *temp, float *humidity = nullptr) = 0;

    protected:
    const char *label_;
    uint16_t interval_;
    float temp_ = INFINITY;
    float humidity_ = -1;

    protected:
    inline void setCachedInfo(float temp, float humidity = -1) {
        temp_ = temp;
        humidity_ = humidity;
    }
    inline void setHumidity(float *humidity, float humidity_ = -1) {
        if (humidity == nullptr) return;
        *humidity = humidity_;
    }
};

template<SensorType Type, class T = void*>
class TypedSensor: public Sensor {
    public:
    TypedSensor(T sensor, const char *label, uint16_t interval): Sensor(label, interval), sensor_(sensor) {}
    TypedSensor(const char *label, uint16_t interval): Sensor(label, interval) {}

    inline SensorType getType() const override final {
        return Type;
    }

    protected:
    T sensor_;
};
\end{lstlisting}

\begin{lstlisting}[style=cpp-small, caption=Клас для роботи з термістром, label=lst:ntc]
template<uint8_t Pin, uint16_t BCoeff, uint8_t NominalTemp = 25>
class NTC {
    public:
    NTC(uint32_t series_res, uint32_t nominal_res = 10000) {
        pinMode(Pin, INPUT);
        series_nominal_ratio_ = (float)series_res/nominal_res;
    }

    static float analogToCelsius(uint16_t analog, float series_nominal_ratio) {
        float temp = series_nominal_ratio * (float)analog / (1023.f - analog);

        temp = log(temp) / (float)BCoeff;
        temp += 1.f /((float)NominalTemp + 273.15);
        return (1.f / temp) - 273.15;
    }

    float getTemperature() {
        uint16_t analog = analogRead(Pin);
        return analogToCelsius(analog, series_nominal_ratio_);
    }

    float getTemperatureAverage(uint8_t samples = 64) {
        uint16_t sum = 0;
        for (uint8_t i = 0; i < samples; ++i) {
            sum += analogRead(Pin);
        }
        return analogToCelsius((float)sum / samples, series_nominal_ratio_);
    }

    private:
    float series_nominal_ratio_ = 1.0;
};
\end{lstlisting}

\begin{lstlisting}[style=cpp-small, caption=Клас-спадок базового класу Sensor для роботи з NTC термістром, label=lst:ntc_sensor]
#include "NTCSensor.h"
#include "Sensor.h"

template<SensorType Type, class T>
class NTCSensor: public TypedSensor<Type, T> {
    public:
    NTCSensor(T sensor, const char *label, uint16_t interval): TypedSensor<Type, T>(sensor, label, interval) {}

    void getInfo(float *temp, float *humidity = nullptr) override {
        *temp = this->sensor_->getTemperatureAverage();
        this->setHumidity(humidity);
        this->setCachedInfo(*temp);
    }
};

#define NTC_PIN A0
#define NTC_R 9975
#define NTC_R1 10000
#define NTC_T 25
#define NTC_B 3950

#define NTC_INTERVAL 500
const char ntc_label[] PROGMEM = "NTC";

using NTCConf = NTC<NTC_PIN, NTC_B, NTC_T>;
NTCConf ntc(NTC_R1, NTC_R);
NTCSensor<SensorType::kNTC, NTCConf*> ntc_sensor(&ntc, ntc_label, NTC_INTERVAL);
\end{lstlisting}

\begin{lstlisting}[style=cpp-small, caption=Клас-спадок базового класу Sensor для роботи з термопарою К-типу, label=lst:tc_sensor]
#include "NTC.h"
#include "Sensor.h"

template<SensorType Type, class T, uint8_t Pin, uint16_t OpAmpGain>
class ThermocoupleSensor: public TypedSensor<Type, T> {
    public:
    ThermocoupleSensor(T sensor, const char *label, uint16_t interval, float seebeck_coeff = 0.0000414): TypedSensor<Type, T>(sensor, label, interval) {
        pinMode(Pin, INPUT);
        seebeck_coeff_ = seebeck_coeff;
    }

    void getInfo(float *temp, float *humidity = nullptr) override {
        getInfo(1, temp, humidity);
    }

    void getInfo(uint8_t samples, float *temp, float *humidity = nullptr) {
        float cold_junction_temp;
        this->sensor_->getInfo(&cold_junction_temp);

        uint16_t analog = 0;
        for (uint8_t i = 0; i < samples; ++i) {
            analog += analogRead(Pin);
        }
        float voltage = (float)analog/samples * (5./1024);
        voltage -= min(voltage, 0.03);
        float new_temp = voltage/((float)OpAmpGain * seebeck_coeff_) + cold_junction_temp;
        *temp = new_temp;

        #ifdef LOG_PLOTTER
        Serial.print("ThermocoupleVoltageX1000:"); Serial.print(voltage*1000, 4); Serial.print(',');
        #endif

        this->setHumidity(humidity);
        this->setCachedInfo(*temp, *humidity);
    }

    bool begin() override {
        return this->sensor_->begin();
    }

    protected:
    float seebeck_coeff_;
};

#define THERM_PIN A3

#ifndef PROTEUS
#define THERM_OP_AMP_GAIN 320
#else
#define THERM_OP_AMP_GAIN 220
#endif

#define THERM_INTERVAL 500

const char thermocouple_label[] PROGMEM = "Thermocouple";

ThermocoupleSensor<SensorType::kThermocouple, Sensor*, THERM_PIN, THERM_OP_AMP_GAIN> thermocouple_sensor(&ntc_sensor, thermocouple_label, THERM_INTERVAL);
\end{lstlisting}

\begin{lstlisting}[style=cpp-small, caption=Клас-спадок базового класу Sensor для роботи з датчиком DS18B20, label=lst:ds18_sensor]
#include <microDS18B20.h>
#include "Sensor.h"

template<SensorType Type, class T>
class DS18Sensor: public TypedSensor<Type, T> {
    public:
    DS18Sensor(T sensor, const char *label, uint16_t interval): TypedSensor<Type, T>(sensor, label, interval) {}

    void getInfo(float *temp, float *humidity = nullptr) override {
        *temp = this->sensor_->readTemp() ? this->sensor_->getTemp() : INFINITY;
        this->setHumidity(humidity);
        this->setCachedInfo(*temp);
    }

    bool begin() override {
        this->sensor_->requestTemp();
        return true;
    }

    bool requestTemp() override {
        this->sensor_->requestTemp();
        return true;
    }
};

#define DS 3
#define DS18_INTERVAL 1000

const char ds18_label[] PROGMEM = "DS18B20";

MicroDS18B20<DS> ds18;

DS18Sensor<SensorType::kDS18, MicroDS18B20<DS>*> ds18_sensor(&ds18, ds18_label, DS18_INTERVAL);
\end{lstlisting}

\begin{lstlisting}[style=cpp-small, caption=Клас-спадок базового класу Sensor для роботи з датчиком DHT11/DHT22, label=lst:dht_sensor]
template<SensorType Type, class T>
class DHTSensor: public TypedSensor<Type, T> {
    public:
    DHTSensor(T sensor, const char *label, uint16_t interval): TypedSensor<Type, T>(sensor, label, interval) {}

    void getInfo(float *temp, float *humidity = nullptr) override {
        sensors_event_t event;
        this->sensor_->temperature().getEvent(&event);
        *temp = event.temperature;
        this->sensor_->humidity().getEvent(&event);
        this->setHumidity(humidity, event.relative_humidity);
        this->setCachedInfo(*temp, *humidity);
    }

    bool begin() override {
        this->sensor_->begin();
        return true;
    }
};

#define DHT11_PIN 10
#define DHT22_PIN 8

#define DHT11_INTERVAL 1000
#define DHT22_INTERVAL 2000

const char dht11_label[] PROGMEM = "DHT11";
const char dht22_label[] PROGMEM = "DHT22";

DHT_Unified dht11(DHT11_PIN, DHT11);
DHT_Unified dht22(DHT22_PIN, DHT22);

DHTSensor<SensorType::kDHT11, DHT_Unified*> dht11_sensor(&dht11, dht11_label, DHT11_INTERVAL);
DHTSensor<SensorType::kDHT22, DHT_Unified*> dht22_sensor(&dht22, dht22_label, DHT22_INTERVAL);
\end{lstlisting}

\begin{lstlisting}[style=cpp-small, caption=Клас-спадок базового класу Sensor для роботи з датчиком HTU21D, label=lst:htu21_sensor]
#include "Sensor.h"

template<SensorType Type, class T>
class HTU21Sensor: public TypedSensor<Type, T> {
    public:
    HTU21Sensor(T sensor, const char *label, uint16_t interval): TypedSensor<Type, T>(sensor, label, interval) {
    }

    void getInfo(float *temp, float *humidity = nullptr) override {
        *temp = this->sensor_->readTemperature();
        this->setHumidity(humidity, this->sensor_->readHumidity());
        this->setCachedInfo(*temp, *humidity);
    }

    bool begin() override {
        return this->sensor_->begin();
    }
};

#define HTU21_INTERVAL 500
const char htu21_label[] PROGMEM = "HTU21D1";

HTU2xD_SHT2x_SI70xx htu21(HTU2xD_SENSOR, HUMD_12BIT_TEMP_14BIT);

HTU21Sensor<SensorType::kHTU21, HTU2xD_SHT2x_SI70xx*> htu21_sensor(&htu21, htu21_label, HTU21_INTERVAL);
\end{lstlisting}

\begin{lstlisting}[style=cpp-small, caption=Код основної частини програми, label=lst:program]
#include <Arduino.h>

#include <Wire.h>

#define LOG_PLOTTER

#include "SensorConfigs/NTC.h"
#include "SensorConfigs/DS18.h"
#include "SensorConfigs/DHT.h"
#include "SensorConfigs/HTU21D1.h"
#include "SensorConfigs/Thermocouple.h"

#include <LiquidCrystal_I2C.h>
#ifdef PROTEUS
#define I2C 0x20
#else
#define I2C 0x27
#endif

LiquidCrystal_I2C lcd(I2C, 16, 2);

#include "LCDSensorPrinter.h"
LCDTempPrinter sensor_printer(&lcd);

#include "Timer.h"
void readTempTimer();

#include "Sensor.h"
#define SENSOR_NUM 6
struct Sensor *sensors[SENSOR_NUM] = {
    &ntc_sensor, &thermocouple_sensor,
    &ds18_sensor, &htu21_sensor,
    &dht11_sensor, &dht22_sensor,
};

int8_t current_sensor_idx = 0;
Sensor *current_sensor = sensors[current_sensor_idx];
Timer<> timer(
    #ifndef LOG_PLOTTER
    current_sensor->getInterval()
    #else
    1000
    #endif
    ,
    readTempTimer);

#include "Button.h"
Button<5, true> button;

void setup() {
    Serial.begin(9600);
    #ifndef PROTEUS
    analogReference(EXTERNAL);
    #endif

    lcd.begin(16, 2);
    lcd.backlight();
    lcd.clear();
    sensor_printer.setLabel(current_sensor->getLabel());
    sensor_printer.setTopArrows();

    for (uint8_t i = 0; i < SENSOR_NUM; i++) {
        sensors[i]->begin();
    }
    Serial.println();
}

void loop() {
    timer.tick();
    button.tick();

    if (button.isClick() || button.isDoubleClick()) {
        static float temp, humidity;

        if (button.isClick()) {
            if (++current_sensor_idx == SENSOR_NUM) {
                current_sensor_idx = 0;
            }
        } else {
            if (--current_sensor_idx == -1) {
                current_sensor_idx = SENSOR_NUM-1;
            }
        }

        current_sensor = sensors[current_sensor_idx];

        current_sensor->getCachedInfo(&temp, &humidity);

        sensor_printer.setLabel(current_sensor->getLabel());
        sensor_printer.setTopArrows();
        sensor_printer.setInfo(temp, humidity);
        button.end();
        #ifndef LOG_PLOTTER
        timer.set_interval(current_sensor->getInterval());
        #endif
    }
}

void readTempTimer() {
    static float temp, humidity;

    #ifdef LOG_PLOTTER
    for (uint8_t i = 0; i < SENSOR_NUM; ++i) {
        sensors[i]->getInfo(&temp, &humidity);
        Serial.print(reinterpret_cast<const __FlashStringHelper*>(sensors[i]->getLabel())); Serial.print(':'); 
        Serial.print(temp); Serial.print(',');
    }
    Serial.println();
    #endif

    current_sensor->getInfo(&temp, &humidity);
    sensor_printer.setInfo(temp, humidity);

    current_sensor->requestTemp();
}
\end{lstlisting}

\begin{lstlisting}[style=cpp-small, caption=Клас для роботи з кнопкою, label=lst:button]
#include "Timer.h"

template<uint8_t Pin, bool Ground = false>
class Button {
    public:
    Button() {
        pinMode(Pin, Ground ? INPUT_PULLUP : INPUT);
    }
    ~Button(){};

    void tick() {
        state_ = Ground ^ digitalRead(Pin);
        if (!is_pressed_ && state_ && timer_.tick()) {
            is_pressed_ = true;
        } else if (is_pressed_ && !state_ && timer_.tick()) {
            is_pressed_ = false;
            if (clicks == 0) {
                first_press_ms = millis();
                is_finished = false;
            }
            ++clicks;
            
            if (clicks < 2) {
                first_press_ms += 400;
            }
        } else if (clicks != 0 && !is_finished && millis() > first_press_ms) {
            is_finished = true;
        } else if (is_finished && millis() - first_press_ms > 500) {
            clicks = 0;
        }
    }

    bool isClick() {
        return is_finished && clicks == 1;
    }

    bool isDoubleClick() {
        return is_finished && clicks == 2;
    }

    void end() {
        if (is_finished) {
            is_finished = false;
            clicks = 0;
        }
    }

    private:
    Timer<> timer_ = Timer<>(100);
    bool state_ = false;
    bool is_pressed_ = false;
    bool is_finished = false;
    uint8_t clicks = 0;
    uint32_t first_press_ms = 0;
};
\end{lstlisting}

\begin{lstlisting}[style=cpp-small, caption=Клас Timer, label=lst:timer]
template<typename Arg = void*>
class Timer {
    public:
    Timer(uint32_t interval, void (*handler)(Arg arg), Arg arg, bool should_start = true): interval_(interval), handler_(handler), arg_(arg) {
        if (should_start) start();
    }
    Timer(uint32_t interval, void (*handler)(), bool should_start = true): interval_(interval) {
        handler_ = (void(*)(void*))(handler);
        if (should_start) start();
    }
    Timer(uint32_t interval, bool should_start = true): Timer(interval, nullptr, should_start) {
    }

    ~Timer() {};

    void start() {
        timer_ = millis();
        running_ = true;
    }

    void stop() {
        timer_ = millis() - timer_;
        running_ = false;
    }

    void resume() {
        timer_ = millis() + timer_;
        running_ = true;
    }

    uint32_t get_interval() {
        return interval_;
    }

    void set_interval(uint32_t interval = 1000) {
        interval_ = interval;
    }

    uint32_t lastTick() {
        return timer_;
    }

    bool isRunning() {
        return running_ && (millis() - timer_) > interval_;
    }

    bool tick() {
        uint32_t now;
        if (running_ && (now = millis()) - timer_ > interval_) {
            timer_ = now;
            if (handler_) handler_(arg_);
            return true;
        }
        return false;
    }

    private:
    uint32_t timer_ = 0;
    uint32_t interval_;
    bool running_ = false;
    void (*handler_)(Arg arg);
    Arg arg_;
};
\end{lstlisting}
