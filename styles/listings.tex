\usepackage{listings}

\lstset{upquote=true}

% straight double quotes hack
\makeatletter
\lst@CCPutMacro
    \lst@ProcessOther {"22}{\lst@ifupquote \textquotedbl
                                     \else \char34\relax \fi}
    \@empty\z@\@empty
\makeatother

% Значения по умолчанию
\lstset{
  basicstyle=\footnotesize,
  breakatwhitespace=true,% разрыв строк только на whitespacce
  breaklines=true,       % переносить длинные строки
%   captionpos=b,          % подписи снизу -- вроде не надо
  inputencoding=koi8-r,
  numbers=left,          % нумерация слева
  numberstyle=\footnotesize,
  showspaces=false,      % показывать пробелы подчеркиваниями -- идиотизм 70-х годов
  showstringspaces=false,
  showtabs=false,        % и табы тоже
  stepnumber=1,
  xleftmargin=2em,
  tabsize=4,              % кому нужны табы по 8 символов?
  frame=single,
  escapeinside={(/}{/)},
  otherkeywords={\&},
  emph={&},
}

% correctly render cyrillic comments
\makeatletter
\lst@InputCatcodes
\def\lst@DefEC{%
    \lst@CCECUse \lst@ProcessLetter
    ^^^^0410^^^^0411^^^^0412^^^^0413^^^^0490^^^^0414^^^^0415^^^^0404% А Б В Г Ґ Д Е Є
    ^^^^0416^^^^0417^^^^0418^^^^0406^^^^0407^^^^0419^^^^041a^^^^041b% Ж З И І Ї Й К Л
    ^^^^041c^^^^041d^^^^041e^^^^041f^^^^0420^^^^0421^^^^0422^^^^0423% М Н О П Р С Т У
    ^^^^0424^^^^0425^^^^0426^^^^0427^^^^0428^^^^0429^^^^042c^^^^042e% Ф Х Ц Ч Ш Щ Ь Ю
    ^^^^042f% Я
    ^^^^0430^^^^0431^^^^0432^^^^0433^^^^0491^^^^0434^^^^0435^^^^0454% а б в г ґ д е є
    ^^^^0436^^^^0437^^^^0438^^^^0456^^^^0457^^^^0439^^^^043a^^^^043b% ж з и і ї й к л
    ^^^^043c^^^^043d^^^^043e^^^^043f^^^^0440^^^^0441^^^^0442^^^^0443% м н о п р с т у
    ^^^^0444^^^^0445^^^^0446^^^^0447^^^^0448^^^^0449^^^^044c^^^^044e% ф х ц ч ш щ ь ю
    ^^^^044f% я
    ^^^^042a^^^^042b^^^^042d% Ъ Ы Э
    ^^^^044a^^^^044b^^^^044d% ъ ы э
    ^^00}
\lst@RestoreCatcodes
\makeatother

% Стиль для псевдокода: строчки обычно короткие, поэтому размер шрифта побольше
\lstdefinestyle{pseudocode}{
  basicstyle=\small,
  keywordstyle=\color{black}\bfseries\underbar,
  language=Pseudocode,
  numberstyle=\footnotesize,
  commentstyle=\footnotesize\it
}

% Стиль для обычного кода: маленький шрифт
\lstdefinestyle{realcode}{
  basicstyle=\scriptsize,
  numberstyle=\footnotesize
}

% Стиль для коротких кусков обычного кода: средний шрифт
\lstdefinestyle{simplecode}{
  basicstyle=\footnotesize,
  numberstyle=\footnotesize
}

\lstdefinestyle{csharp}{language=[Sharp]C, style=simplecode}
\lstdefinestyle{csharp-small}{language=[Sharp]C, style=realcode}
\lstdefinestyle{cpp-small}{language=C++, style=realcode}
