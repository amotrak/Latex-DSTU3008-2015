\Introduction

У сучасному світі температурні датчики використовуються майже у будь-яких пристроях, приладах та побутових речах, якими ми користуємося кожен день. Їхня роль у сучасній технологічній інфраструктурі стала невід'ємною, оскільки вимірювання температури є ключовим аспектом в ряді сфер, включаючи промисловість, науку, медицину та побутове використання. З точністю та надійністю цих датчиків пов'язана ефективність та якість багатьох технічних систем. 

Наприклад, у важкої промисловості вони використовуються в екстремальних умовах, як основний елемент термостату для контролю температури плавлення металів, наприклад сталі $\approx\ignorespaces1500^\circ C$. Іншим прикладом є їх використання в автомобільній промисловості, де температурні датчики вимірюють температуру охолоджувальної рідини або масла, що дозволяє системі керування двигуном забезпечити оптимальні умови для роботи автомобіля. У побуті температурні датчики контролюють температуру в приміщеннях та регулюють роботу систем опалення та кондиціонування.

Щоб зрозуміти, важливість датчиків температури, не потрібно навіть заходити в інтернет, можна просто порахувати їх кількість у квартирі: два датчику у холодильник -- основна та морозильні камери; газова плита, яка підтримує налаштовану температуру у духовці, приблизно $\approx\ignorespaces270^\circ~C$; термометри у годиннику, на балконі; у ванній кімнати бойлер та пральна машина, в обох випадках використовується терморегулятор для нагріву ТЕНа до потрібної температури; у телефоні AIDA64 показала наявність аж 18 датчиків; роутери, телевізор, паяльна станція...

Досить цікавим є той факт, що температура з різних типів датчиків буде відрізнятися на $\ignorespaces 0.5^\circ~C$ -- $\ignorespaces 1^\circ~C$, але це не буде означати, що один з датчиків працює ``правильно'', а інші з похибкою.

У курсовому проєкті буде зібрано пристрій, на базі Arduino UNO R3, до якого під'єднаємо декілька найцікавіших, на мій погляд, датчиків. Уся інформація буде виводитися на LCD дисплей, дані для графіків будемо відправляти за допомогою Serial інтерфейсу на комп'ютер, де далі будувати та порівнювати графіки.
\clearpage
